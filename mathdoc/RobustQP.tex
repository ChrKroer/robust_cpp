\documentclass[11pt,oneside]{article}

\usepackage{amsthm,amsmath,amssymb,amsfonts}
\usepackage{mathtools,xfrac}
\usepackage{url,setspace}
\usepackage{subfigure,graphicx,epsfig,tikz,caption}
\usepackage{xcolor}
\definecolor{olive}{rgb}{0.3, 0.4, .1}
\definecolor{fore}{RGB}{249,242,215}
\definecolor{back}{RGB}{51,51,51}
\definecolor{title}{RGB}{255,0,90}
\definecolor{dgreen}{rgb}{0.,0.6,0.}
\definecolor{gold}{rgb}{1.,0.84,0.}
\definecolor{JungleGreen}{cmyk}{0.99,0,0.52,0}
\definecolor{BlueGreen}{cmyk}{0.85,0,0.33,0}
\definecolor{RawSienna}{cmyk}{0,0.72,1,0.45}
\definecolor{Magenta}{cmyk}{0,1,0,0}
\def\cbl{\color{blue}}
%--tikz defs
\colorlet{circle edge}{blue!50}
\colorlet{circle area}{blue!20}
\tikzset{filled/.style={fill=circle area, draw=circle edge, thick},
    outline/.style={draw=circle edge, thick}}

\usepackage{enumerate}
\usepackage{pdfsync,array}
\usepackage[normalem]{ulem}

\usepackage[ruled]{algorithm2e}
\renewcommand{\algorithmcfname}{ALGORITHM}
\SetAlFnt{\small}
\SetAlCapFnt{\small}
\SetAlCapNameFnt{\small}
\SetAlCapHSkip{0pt}
\IncMargin{-\parindent}

\usepackage{verbatim}
\usepackage{makeidx,latexsym}
\usepackage{bm}
\usepackage{authblk}
\usepackage[margin=2.8cm]{geometry}
\usepackage[colorlinks=true]{hyperref}

\usepackage[numbers,sort&compress,square,comma]{natbib}
\usepackage{authblk}
\usepackage[in]{fullpage}
\makeatletter
\def\imod#1{\allowbreak\mkern10mu({\operator@font mod}\,\,#1)}
\makeatother
\usepackage[compact]{titlesec}


%\usepackage{graphicx}
%\usepackage{subfig}


%Colors
\def\crd{\color{red}}
\def\cbl{\color{blue}}
\def\cgr{\color{green}}
\def\cmg{\color{magenta}}
\def\ccy{\color{cyan}}
\def\cye{\color{yellow}}
\definecolor{gold}{rgb}{0.85,0.65,0}
%\colorlet{TepperRed}{red!70!black} 

%Shorthand
\newcommand{\wrt}{w.r.t.}

%Equations, numbering, etc
\newcommand{\nin}{\noindent}
\newcommand{\nn}{\nonumber \\}
\renewcommand{\theequation}{\mbox{\arabic{equation}}}
\newcommand{\be}{\begin{eqnarray}}
\newcommand{\ee}[1]{\label{#1}\end{eqnarray}}
\newcommand{\ese}{\end{eqnarray*}}
\newcommand{\bse}{\begin{eqnarray*}}
\def\beq{\begin{equation}}
\def\eeq{\end{equation}}
\def\Bbb#1{{\bf #1}}
\def\eqnum#1{\eqno (#1)}
\def\fnote#1{\footnote}
%\newcommand{\epr}{\hfill\ensuremath{\Box}}
\newcommand{\epr}{\hfill\hbox{\hskip 4pt \vrule width 5pt height 6pt depth 1.5pt}\vspace{0.0cm}\par}
%\def\qed{$\Box$}
\def\qed{\ \hfill$\square$\par\smallskip}


\def\ra{\rangle}
\def\la{\langle}
\def\ul{\underline}
%\def\*{{{\LARGE\bf $^*$}}}
%\def\P{{{\LARGE\bf $^+$}}}
%\def\PP{{{\LARGE\bf $^{++}$}}}

\newcommand{\norm}[1]{\left\Vert #1 \right\Vert}

\newcommand{\grad}{\ensuremath{\nabla}}

%P, NP, co-NP, etc.
%\def\P{{\textsf\bf P\,}}
%\def\NP{{\textsf\bf NP\,}}


%Letters in different typefaces
%\newcommand{\A}{\mathbb{A}}
\def\A{{\mathbb{A}}}
\def\B{{\mathbb{B}}}
\def\C{{\mathbb{C}}}
\def\D{{\mathbb{D}}}
\def\E{{\mathbb{E}}}
\def\F{{\mathbb{F}}}
\def\G{{\mathbb{G}}}
\def\H{{\mathbb{H}}}
\def\I{{\mathbb{I}}}
\def\J{{\mathbb{J}}}
\def\K{{\mathbb{K}}}
\def\L{{\mathbb{L}}}
\def\M{{\mathbb{M}}}
\def\N{{\mathbb{N}}}
\def\O{{\mathbb{O}}}
\def\P{{\mathbb{P}}}
\def\Q{{\mathbb{Q}}}
\def\R{{\mathbb{R}}}
\def\S{{\mathbb{S}}}
\def\T{{\mathbb{T}}}
\def\U{{\mathbb{U}}}
\def\V{{\mathbb{V}}}
\def\W{{\mathbb{W}}}
\def\X{{\mathbb{X}}}
\def\Y{{\mathbb{Y}}}
\def\Z{{\mathbb{Z}}}
\def\Le{{\mathbb{L}}}
\def\Se{{\mathbb{S}}}


\def\bA{{\mathbf{A}}}
\def\bB{{\mathbf{B}}}
\def\bC{{\mathbf{C}}}
\def\bD{{\mathbf{D}}}
\def\bE{{\mathbf{E}}}
\def\bF{{\mathbf{F}}}
\def\bG{{\mathbf{G}}}
\def\bH{{\mathbf{H}}}
\def\bI{{\mathbf{I}}}
\def\bJ{{\mathbf{J}}}
\def\bK{{\mathbf{K}}}
\def\bL{{\mathbf{L}}}
\def\bM{{\mathbf{M}}}
\def\bN{{\mathbf{N}}}
\def\bO{{\mathbf{O}}}
\def\bP{{\mathbf{P}}}
\def\bQ{{\mathbf{Q}}}
\def\bR{{\mathbf{R}}}
\def\bS{{\mathbf{S}}}
\def\bT{{\mathbf{T}}}
\def\bU{{\mathbf{U}}}
\def\bV{{\mathbf{V}}}
\def\bW{{\mathbf{W}}}
\def\bX{{\mathbf{X}}}
\def\bY{{\mathbf{Y}}}
\def\bZ{{\mathbf{Z}}}

\def\cA{\mathcal{A}}
\def\cB{{\cal B}}
\def\cC{\mathcal{C}}
\def\cD{\mathcal{D}}
\def\cE{{\cal E}}
\def\cF{{\cal F}}
\def\cG{{\cal G}}
\def\cH{{\cal H}}
\def\cI{{\cal I}}
\def\cJ{{\cal J}}
\def\cK{{\cal K}}
\def\cL{{\cal L}}
\def\cM{{\cal M}}
\def\cN{{\cal N}}
\def\cO{{\cal O}}
\def\cP{{\cal P}}
\def\cQ{{\cal Q}}
\def\cR{{\cal R}}
\def\cS{\mathcal{S}}
\def\cT{{\cal T}}
\def\cU{{\cal U}}
\def\cV{\mathcal{V}}
\def\cW{{\cal W}}
\def\cX{{\cal X}}
\def\cY{{\cal Y}}
\def\cZ{\mathcal{Z}}

\newcommand{\bbA}{\mathbb{A}}
\newcommand{\bbB}{\mathbb{B}}
\newcommand{\bbC}{\mathbb{C}}
\newcommand{\bbD}{\mathbb{D}}
\newcommand{\bbE}{\mathbb{E}}
\newcommand{\bbF}{\mathbb{F}}
\newcommand{\bbG}{\mathbb{G}}
\newcommand{\bbH}{\mathbb{H}}
\newcommand{\bbI}{\mathbb{I}}
\newcommand{\bbJ}{\mathbb{J}}
\newcommand{\bbK}{\mathbb{K}}
\newcommand{\bbL}{\mathbb{L}}
\newcommand{\bbM}{\mathbb{M}}
\newcommand{\bbN}{\mathbb{N}}
\newcommand{\bbO}{\mathbb{O}}
\newcommand{\bbP}{\mathbb{P}}
\newcommand{\bbQ}{\mathbb{Q}}
\newcommand{\bbR}{\mathbb{R}}
\newcommand{\bbS}{\mathbb{S}}
\newcommand{\bbT}{\mathbb{T}}
\newcommand{\bbU}{\mathbb{U}}
\newcommand{\bbV}{\mathbb{V}}
\newcommand{\bbW}{\mathbb{W}}
\newcommand{\bbX}{\mathbb{X}}
\newcommand{\bbY}{\mathbb{Y}}
\newcommand{\bbZ}{\mathbb{Z}}

\newcommand{\CA}{\mathcal{A}}
\newcommand{\CB}{\mathcal{B}}
\newcommand{\CC}{\mathcal{C}}
\newcommand{\CD}{\mathcal{D}}
\newcommand{\CE}{\mathcal{E}}
\newcommand{\CF}{\mathcal{F}}
\newcommand{\CG}{\mathcal{G}}
\newcommand{\CH}{\mathcal{H}}
\newcommand{\CI}{\mathcal{I}}
\newcommand{\CJ}{\mathcal{J}}
\newcommand{\CK}{\mathcal{K}}
\newcommand{\CL}{\mathcal{L}}
\newcommand{\CM}{\mathcal{M}}
\newcommand{\CN}{\mathcal{N}}
\newcommand{\CO}{\mathcal{O}}
\newcommand{\CP}{\mathcal{P}}
\newcommand{\CQ}{\mathcal{Q}}
\newcommand{\CR}{\mathcal{R}}
\newcommand{\CS}{\mathcal{S}}
\newcommand{\CT}{\mathcal{T}}
\newcommand{\CU}{\mathcal{U}}
\newcommand{\CV}{\mathcal{V}}
\newcommand{\CW}{\mathcal{W}}
\newcommand{\CX}{\mathcal{X}}
\newcommand{\CY}{\mathcal{Y}}
\newcommand{\CZ}{\mathcal{Z}}



\def\mones{{\mathbf{1}}}

\newcommand{\ihat}{\hat{\imath}}
\newcommand{\jhat}{\hat{\jmath}}

\def\Ic{{\overline{I}}}

\def\e{\mathop{{\rm e}}}
%\def\i{{\mathop{z}}}
\def\nrm{\mathop{{\parallel}}}
\def\vrt{\mathop{\,\mid}\,}
\def\eqbd{\colon =}

%Optimization basics
\DeclareMathOperator{\Opt}{Opt}
\def\Argmin{\mathop{\rm Argmin}}
%\def\argmin{\mathop{\rm argmin}}
\def\Argmax{\mathop{\rm Argmax}}
%\def\argmax{{\mathop{\rm argmax}}}
\DeclareMathOperator*{\argmin}{arg\,min}
\DeclareMathOperator*{\argmax}{arg\,max}

%Saddle point and VI related
\DeclareMathOperator{\SadVal}{SV}
\DeclareMathOperator{\SV}{SV}
\DeclareMathOperator{\sad}{sad}
\newcommand{\VI}{VI}
\DeclareMathOperator{\Proj}{Proj}
\DeclareMathOperator{\Prox}{Prox}
\DeclareMathOperator{\Entr}{Entr}
%\def\epsilonsad{{\epsilon_{\hbox{\scriptsize\rm sad}}}}
\def\epsilonsad{\epsilon_{\sad}}
\def\epsilonvi{\epsilon_{\hbox{\scriptsize VI}}}
\def\epsilonN{\epsilon_{\hbox{\scriptsize Nash}}}
\def\Sad{{\mathop{\rm Sad}}}
\def\Residual{\mathop{{\rm Residual}}}
\def\DualityGap{{\hbox{\rm DualityGap}}}
\newcommand{\dgf}{d.g.f.}



\DeclareMathOperator{\sign}{sign}
\DeclareMathOperator{\sgn}{sgn}
\DeclareMathOperator{\Supp}{supp}
\DeclareMathOperator{\supp}{supp}
\DeclareMathOperator{\Card}{Card}
\DeclareMathOperator{\card}{card}
\DeclareMathOperator{\Size}{Size}
\DeclareMathOperator{\size}{Size}
\DeclareMathOperator{\Sz}{Size}
\DeclareMathOperator{\Vol}{Vol}
\DeclareMathOperator{\vol}{Vol}
\DeclareMathOperator{\Diam}{Diam}
\DeclareMathOperator{\AvDiam}{AvDiam}
\def\const{{\hbox{\rm const}}}
\def\data{\hbox{\rm data}}
\def\nnz{{\hbox{\rm nnz}}}

%Matrix related
\DeclareMathOperator{\rank}{rank}
\DeclareMathOperator{\Diag}{Diag}
\DeclareMathOperator{\diag}{diag}
\DeclareMathOperator{\Trace}{Trace}
\DeclareMathOperator{\Tr}{Tr}
\DeclareMathOperator{\Det}{Det}
%\DeclareMathOperator{\det}{det}
\DeclareMathOperator{\Col}{Col}
\DeclareMathOperator{\Row}{Row}
\DeclareMathOperator{\Null}{Null}
%\DeclareMathOperator{\Ker}{Ker} %defined later in the "function related" section
\DeclareMathOperator{\Image}{Im}
\DeclareMathOperator{\Imm}{Im}
\DeclareMathOperator{\Range}{Range}
\def\Arrow{{\hbox{\rm Arrow}}}
\def\Fr{{\hbox{\scriptsize Fr}}}
\def\Fro{\mathop{\hbox{\rm\scriptsize Fro}}}
\def\Spec{\mathop{\hbox{\rm\scriptsize Spec}}}


%Probability and distributions
\def\Prob{{\hbox{\rm Prob}}}
\def\Cov{{\hbox{\rm Cov}}}
\def\uniform{\mathop{\rm uniform}}
\DeclareMathOperator{\Uniform}{Uniform}
\DeclareMathOperator{\Geometric}{Geometric}
\DeclareMathOperator{\Poisson}{Poisson}
\DeclareMathOperator{\Var}{Var} %for variance
\DeclareMathOperator{\CVaR}{CVaR}
\DeclareMathOperator{\KL}{KL}


%Topology related
\def\dim{\mathop{{\rm dim}\,}}
%\DeclareMathOperator{\dim}{dim}
\DeclareMathOperator{\codim}{codim}
\DeclareMathOperator{\Lin}{Lin}
\DeclareMathOperator{\lin}{lin}
\DeclareMathOperator{\Aff}{Aff}
\DeclareMathOperator{\aff}{aff}
\DeclareMathOperator{\Span}{Span}
\DeclareMathOperator{\intt}{int}
\DeclareMathOperator{\myint}{int}
\DeclareMathOperator{\inter}{int}
\DeclareMathOperator{\relint}{rel int} %rint %ri
\DeclareMathOperator{\rint}{rel int}
\DeclareMathOperator{\ri}{rel int}
\DeclareMathOperator{\partialri}{rel int}
%\def\partialri{\mathop{\partial_{\rm ri}}}
\DeclareMathOperator{\cl}{cl}
%\def\clsr{{\rm cl}}
\DeclareMathOperator{\bd}{bd}
\DeclareMathOperator{\rbd}{rbd}

%Function related
\DeclareMathOperator{\Dom}{Dom}
\DeclareMathOperator{\dom}{dom}
\DeclareMathOperator{\Epi}{Epi}
\DeclareMathOperator{\epi}{epi}
\DeclareMathOperator{\Ker}{Ker}
%\DeclareMathOperator{\ker}{Ker}

%Convexity related
\DeclareMathOperator{\Conv}{Conv}
\DeclareMathOperator{\conv}{conv}
\DeclareMathOperator{\co}{conv}
\DeclareMathOperator{\Cone}{Cone}
\DeclareMathOperator{\cone}{cone}
%\def\clconv{\mathop{{\rm \overline{conv}}}}
\DeclareMathOperator{\clconv}{\overline{conv}}
\DeclareMathOperator{\clcone}{\overline{cone}}
\DeclareMathOperator{\ccvh}{\overline{conv}}
\DeclareMathOperator{\ccnh}{\overline{cone}}
\def\Ext{{\mathop{\rm Ext}}}
\def\Extr{{\mathop{\rm Extr}}}
\DeclareMathOperator{\Rec}{Rec}
\DeclareMathOperator{\rec}{Rec}
\DeclareMathOperator{\apex}{apex}


\def\dist{\mathop{{\rm dist}}}
\def\log{\mathop{{\rm log}}}
\def\asin{{\hbox{\rm asin}}}
\def\Polar{{\mathop{{\rm Polar\,}}}}
\def\ch{{\mathop{\rm ch\,}}}
\def\arc{{\mathop{\rm arc\,}}}
\def\EllIn{{\mathop{\rm EllIn}}}
\def\GDA{{\rm \ ArD\ }}
\def\GDS{{\rm \ StD\ }}
\def\lev{{\mathop{\rm lev}}}


%Shapes etc.
\def\rdiamondsuit{{\color{red}$\diamondsuit$}}
\def\rclubsuit{{\color{red}$\clubsuit$}\ }
\def\rspadesuit{{\color{red}$\spadesuit$}\ }
\def\rbullet{{\color{red}$\bullet$}\ }
\def\rheartsuit{{\color{red}$\heartsuit$}}
\def\rArrow{{\color{red}$\Rightarrow$}\ }
\def\blacksquare{\hbox{\vrule width 4pt height 4pt depth 0pt}}
\def\square{\hbox{\vrule\vbox{\hrule\phantom{o}\hrule}\vrule}}


%Spacing, picture sizing related
\def\nr{\par \noindent}
\def\pn{\par\noindent}
\def\qr{\qquad\cr}
\def\fig{\epsfxsize=220pt\epsfysize=140pt\epsffile}
\def\mypictsmall{\epsfxsize=120pt\epsfysize=120pt\epsffile}
\def\mypictvsmall{\epsfxsize=90pt\epsfysize=90pt\epsffile}

\def\huh{\hbox{\vrule width 2pt height 8pt depth 2pt}}



\def\CE{\mathop{{\hbox{\sl CircEll}}}}
\def\CS{\mathop{{\hbox{\sl CircSimp}}}}
\def\IE{\mathop{{\hbox{\sl InscrEll}}}}
\def\Compl{\mathop{{\rm Compl}}}
\def\Accur{\mathop{{\rm Accur}}}
\def\mes{\mathop{{\rm mes}}}
\def\EllOut{\mathop{{\rm EllOut}}}
\def\Ell{\mathop{{\sl Ell}}}
\def\InnerRad{\mathop{{\rm InnerRad}}}
\def\Mom{\mathop{\rm Mom}}
\def\Loss{{\hbox{\rm Loss}}}


\def\bStnu{\bS^{\widetilde{\nu}}}
\def\bStnup{\bS^{\widetilde{\nu}}_+}






\theoremstyle{plain}
\newtheorem{theorem}{Theorem}[section]
\newtheorem{conjecture}{Conjecture}[section]
\newtheorem{lemma}[theorem]{Lemma}
\newtheorem*{lemma*}{Lemma}
\newtheorem{proposition}[theorem]{Proposition}
\newtheorem*{proposition*}{Proposition}
\newtheorem{corollary}[theorem]{Corollary}

\theoremstyle{definition}
\newtheorem{question}{Question}[section]
\newtheorem{definition}{Definition}[section]
\newtheorem{assumption}{Assumption}[section]
\newtheorem{condition}{Condition}[section]
\newtheorem{observation}[theorem]{Observation}
\newtheorem*{observation*}{Observation}

\theoremstyle{remark}
\newtheorem{remark}{Remark}[section]
\newtheorem*{remark*}{Remark}
\newtheorem{example}[theorem]{Example}
\newtheorem*{example*}{Example}

\DeclareMathOperator{\train}{train}
\DeclareMathOperator{\test}{test}
\DeclareMathOperator{\MAE}{MAE}


\title{Robust Quadratic Programming}
\author[1]{Nam Ho-Nguyen}
%\affil[1]{Tepper School of Business, Carnegie Mellon University, Pittsburgh, PA, 15213, USA.}
\date{}%\date{February 19, 2017}

\begin{document}
\maketitle

There is a summary for computing subgradients of robust quadratic constraints at the bottom if you want to skip the details.

\section{Robust Portfolio Optimization}

Given $n$ assets, let:
\begin{itemize}
\item $r \in \bbR^n$ be a vector of returns for each asset.
\item $\mu \in \bbR^n$ be a vector of mean returns for each asset.
\item $f \in \bbR^m$ be factors for the market.
\item $V \in \bbR^{m \times n}$ be a matrix of factor loadings.
\item $\epsilon \in \bbR^m$ be a vector of residual errors for each asset.
\end{itemize}
The factor model specifies that
\[ r = \mu + V^\top f + \epsilon. \]
In addition, we assume that $f$ and $\epsilon$ are independently distributed, with
\begin{itemize}
\item $\bbE[f]=\bm{0}_m$, $\Cov(f) = \bbE[ff^\top] = F$ for a given positive definite $F \in \bbR^{m \times m}$.
\item $\bbE[\epsilon]=\bm{0}_n$, $\Cov(\epsilon) = \bbE[\epsilon \epsilon^\top] = D$ for a given positive definite diagonal $D \in \bbR^{n \times n}$.
\end{itemize}
Then $\bbE[r] = \mu$, $\Cov(r) = \bbE[(r-\mu)(r-\mu)^\top] = V^\top F V + D$. Let $x \in \bbR^n$ be an allocation of investments in each asset. We allow shorting, so we need not restrict $x$ to be non-negative, but we do need our total investments not to exceed our wealth, and after normalising by initial wealth, we need the constraint $\bm{1}_n^\top x = 1$. Our return from $x$ is $r^\top x$. We can easily check that
\[ \bbE[r^\top x] = \mu^\top x, \quad \Cov(r^\top x) = \bbE[((r-\mu)^\top x)^2] = x^\top (V^\top F V + D) x. \]

The Markowitz mean-variance portfolio is the one that minimizes the variance (or risk) while maximizing the mean. In practice, we don't know $\mu$ and $V$ a priori, so we estimate them via linear regression from past data on $r$ and $f$ to obtain $\mu_0$ and $V_0$. The portfolio optimization problem is
\begin{align*}
\min &\quad b + c - \lambda a\\
\text{s.t.} &\quad \mu_0^\top x \geq a\\
&\quad x^\top (V_0^\top F V_0) x \leq b\\
&\quad x^\top D x \leq c\\
&\quad \bm{1}_n^\top x = 1.
\end{align*}
Here, $\lambda \geq 0$ is a fixed parameter that trades off minimizing risk and maximizing return.

Since we learn $\mu_0, V_0$ from a regression, they are uncertain. We use robust optimization to immunize our solutions against this uncertainty. Specifically, we will define two uncertainty sets $U_\mu$, $U_V$ and instead solve
\begin{align*}
\min &\quad b + c - \lambda a\\
\text{s.t.} &\quad \min_{\tilde{\mu} \in U_\mu} \tilde{\mu}^\top x \geq a\\
&\quad \max_{\tilde{V} \in U_V} x^\top (\tilde{V}^\top F \tilde{V}) x \leq b\\
&\quad x^\top D x \leq c\\
&\quad \bm{1}_n^\top x = 1.
\end{align*}

For $\mu_0$, we define
\begin{align*}
U_\mu &= \{ \mu_0 + u : u \in \bbR^n,\ |u_i| \leq \gamma_i, \ i \in [n] \}
\end{align*}
where $\gamma_i$ are fixed constants defined from the regression output. Note that
\[ \min_{\tilde{\mu} \in U_\mu} \tilde{\mu}^\top x \geq \alpha \iff \mu_0^\top x - \sum_{i \in [n]} \gamma_i |x_i| \geq \alpha. \]
This can be recast as a linear system
\begin{align*}
\mu_0^\top x  &\geq \alpha + \sum_{i \in [n]} \gamma_i w_i\\
w_i &\geq x_i, \quad i \in [n]\\
w_i &\geq -x_i, \quad i \in [n].
\end{align*}
For $V_0$, we will define the uncertainty set
\[ U_V = \left\{ V_0 + \sum_{j \in [k]} P_j u_j : u \in \bbR^k,\ \|u\|_2 \leq 1 \right\} \]
where $P_j \in \bbR^{m \times n}$ are fixed matrices.



\section{Robust SVM}

We are given data $\{x_i,y_i\}_{i \in [m]}$ where $x_i \in \bbR^n$ and $y_i \in \{\pm 1\}$. We wish to learn a classifier $c(x) = \sign(\la w,x \ra + b)$. The parameters $w,b$ will be learnt from data by minimizing the hinge loss $h(r) = \max\{1-r,0\}$ between points on the hyperplane $\la w,x_i \ra + b$ and labels $y_i$, specifically, we want to solve
\begin{align*}
\min_{a,w,b} &\quad \sum_{i \in [m]} a_i\\
\text{s.t.} &\quad a_i \geq 1 - y_i(\la w,x_i \ra + b), \quad i \in [m]\\
&\quad a_i \geq 0, \quad i \in [m].
\end{align*}
We also often add regularization on $w$ to ensure well-posedness of the problem:
\begin{align*}
\min_{a,w,b} &\quad C \sum_{i \in [m]} a_i + \frac{1}{2} \|w\|_2^2\\
\text{s.t.} &\quad a_i \geq 1 - y_i(\la w,x_i \ra + b), \quad i \in [m]\\
&\quad a_i \geq 0, \quad i \in [m],
\end{align*}
where $C > 0$ is the regularization parameter. We rewrite this as as a second-order cone program:
\begin{align*}
\min_{a,w,b} &\quad C \sum_{i \in [m]} a_i + \frac{1}{2} t\\
\text{s.t.} &\quad a_i \geq 1 - y_i(\la w,x_i \ra + b), \quad i \in [m]\\
&\quad a_i \geq 0, \quad i \in [m]\\
&\quad 1+t \geq \sqrt{4 \|w\|_2^2 + (1-t)^2}.
\end{align*}
After some straightforward reasoning, the dual of this SOCP is equivalent to
\begin{align*}
\min_{\gamma,\alpha} &\quad \frac{1}{2} \gamma - \sum_{i \in [m]} \alpha_i\\
\text{s.t.} &\quad 0 \leq \alpha_i \leq C, \quad i \in [m]\\
&\quad \sum_{i \in [m]} y_i \alpha_i = 0\\
&\quad \alpha^\top Y^\top X^\top X Y \alpha \leq \gamma,
\end{align*}
where $X = \begin{bmatrix} x_1 & \ldots & x_m \end{bmatrix} \in \bbR^{n \times m}$, and $Y = \Diag(\{y_i\}_{i \in [m]}) \in \bbR^{m \times m}$.

If the data points $x_i$ are noisy, then we observe $x_{0,i} \approx x_i$. Furthermore, our true data matrix $X$ is now $X = X_0 + W$. Since we don't know what $W$ will be, we use a robust constraint where we optimize for all $X \in U_X$. We define the uncertainty set similar to before:
\[U_X = \left\{ X_0 + \sum_{j \in [k]} P_j u_j : u \in \bbR^k, \ \|u\|_2 \leq 1 \right\},\]
where $P_j \in \bbR^{n \times m}$ are fixed matrices. The dual problem for SVM becomes
\begin{align*}
\min_{\gamma,\alpha} &\quad \frac{1}{2} \gamma - \sum_{i \in [m]} \alpha_i\\
\text{s.t.} &\quad 0 \leq \alpha_i \leq C, \quad i \in [m]\\
&\quad \sum_{i \in [m]} \alpha_i y_i = 0\\
&\quad \max_{\|u\|_2 \leq 1} \alpha^\top Y^\top \left(X_0 + \sum_{j \in [k]} P_j u_j\right)^\top \left(X_0 + \sum_{j \in [k]} P_j u_j\right) Y \alpha \leq \gamma.
\end{align*}
This robust quadratic constraint is of the same form as the ones from the robust portfolio optimization problem. We examine how to process these next.

\section{Robust Convex Quadratic Constraints}

We now examine a certain type of uncertainty on convex quadratic constraints of the form
\[ \left\| Ax \right\|_2^2 = x^\top A^\top A x \leq c, \]
where $x \in \bbR^n$, $A \in \bbR^{m \times n}$, and $c \in \bbR$. We can also include a linear term $b^\top x$ but this will not affect things much, so for simplicity we exclude it. The uncertainty we consider is on the $A$ matrix, and takes the form
\[ U_A = \left\{ A_0 + \sum_{j \in [k]} P_j u_j : u \in \bbR^k,\ \|u\|_2 \leq 1 \right\}, \]
for fixed matrices $P_j \in \bbR^{m \times n}$, $j \in [k]$. For $A \in U_A$, rewrite
\begin{align*}
x^\top A^\top A x &= x^\top \left( A_0 + \sum_{j \in [k]} P_j u_j \right)^\top \left( A_0 + \sum_{j \in [k]} P_j u_j \right) x\\
&= x^\top A_0^\top A_0 x + 2 \sum_{j \in [k]} (x^\top A_0^\top P_j x) u_j + x^\top \left( \sum_{j \in [k]} P_j u_j \right)^\top \left( \sum_{j \in [k]} P_j u_j \right) x\\
&= x^\top A_0^\top A_0 x + 2 \sum_{j \in [k]} (x^\top A_0^\top P_j x) u_j + \sum_{j,j' \in [k]} \left(x^\top P_j^\top P_{j'} x\right) u_j u_{j'}.
\end{align*}
We know that this is convex in $x$ since $A^\top A$ is always positive semidefinite. In fact, it is also convex in $u$. To see this, let $Y(x) \in \bbR^{k \times k}$ be the matrix with entries $Y(x)_{jj'} = x^\top P_j^\top P_{j'} x$. Then in fact $Y(x)$ is positive semidefinite, since we can write
\[ Y(x) = \begin{bmatrix}
x^\top P_1^\top\\
\vdots\\
x^\top P_k^\top
\end{bmatrix}
\underbrace{\begin{bmatrix}
P_1 x & \ldots & P_k x
\end{bmatrix}}_{=Y'(x)} = Y'(x)^\top Y'(x).\]
We can now write
\begin{align*}
&x^\top \left( A_0 + \sum_{j \in [k]} P_j u_j \right)^\top \left( A_0 + \sum_{j \in [k]} P_j u_j \right) x\\
&= x^\top A_0^\top A_0 x + 2 \sum_{j \in [k]} (x^\top A_0^\top P_j x) u_j + u^\top Y(x) u.
\end{align*}
The first term is constant in $u$, the second term is linear, and the third term is a convex quadratic in $u$. To satisfy the robust constraint, we need
\[ \max_{A \in U_A} x^\top A^\top A x = \max_{\|u\|_2 \leq 1} \left\{ x^\top A_0^\top A_0 x + 2 \sum_{j \in [k]} (x^\top A_0^\top P_j x) u_j + u^\top Y(x) u \right\} \leq c. \]
Thus, a pessimization oracle for this constraint needs to maximize a convex quadratic in $u$, which is a trust region subproblem (TRS). We can make the TRS objective concave by shifting the $Y(x)$ matrix (letting $I_k$ be the $k \times k$ identity matrix):
\begin{align*}
&\max_{\|u\|_2 \leq 1} \left\{ x^\top A_0^\top A_0 x + 2 \sum_{j \in [k]} (x^\top A_0^\top P_j x) u_j + u^\top Y(x) u \right\}\\
&= \max_{\|u\|_2 \leq 1} \left\{ x^\top A_0^\top A_0 x + 2 \sum_{j \in [k]} (x^\top A_0^\top P_j x) u_j + u^\top (Y(x) - \lambda_{\max}(Y(x)) I_k) u + \lambda_{\max}(Y(x)) \right\}\\
&= \max_{\|u\|_2 \leq 1} \left\{ x^\top A_0^\top A_0 x + 2 \sum_{j \in [k]} (x^\top A_0^\top P_j x) u_j + u^\top Y(x) u + \lambda_{\max}(Y(x))(1 - \|u\|_2^2) \right\}.
\end{align*}
For a pessimization oracle, we can just input this into Gurobi, since the new objective is a concave quadratic. Note that this new quadratic is still convex in $x$, as we will explain later.

Let us now examine this new objective within a subgradient scheme. Define
\[ f(x,u) = x^\top A_0^\top A_0 x + 2 \sum_{j \in [k]} (x^\top A_0^\top P_j x) u_j + u^\top Y(x) u + \lambda_{\max}(Y(x))(1 - \|u\|_2^2). \]
We are interested in $\grad_x f(x,u)$ and $\grad_u f(x,u)$. The gradient in $u$ is straightforward:
\[ \grad_u f(x,u) = 2 \{x^\top A_0^\top P_j x\}_{j \in [k]} + 2 Y(x) u - 2 \lambda_{\max}(Y(x)) u. \]
The gradient in $x$ is a bit more complicated. It helps to rewrite $f(x)$ using the original objective:
\[ f(x,u) = x^\top \left( A_0 + \sum_{j \in [k]} P_j u_j \right)^\top \left( A_0 + \sum_{j \in [k]} P_j u_j \right) x + \lambda_{\max}(Y(x))(1 - \|u\|_2^2). \]
Then
\[ \grad_x f(x,u) = 2 \left( A_0 + \sum_{j \in [k]} P_j u_j \right)^\top \left( A_0 + \sum_{j \in [k]} P_j u_j \right) x + (1 - \|u\|_2^2) \grad_x \lambda_{\max}(Y(x)). \]
The term we need to investigate further is $\grad_x \lambda_{\max}(Y(x))$. To do this, we rewrite
\begin{align*}
\lambda_{\max}(Y(x)) &= \max_{\|v\|_2 \leq 1} v^\top Y(x) v = \max_{\|v\|_2 \leq 1} \sum_{j,j' \in [k]} \left(x^\top P_j^\top P_{j'} x\right) v_j v_{j'}\\
&= \max_{\|v\|_2 \leq 1} x^\top \left( \sum_{j \in [k]} P_j v_j \right)^\top \left( \sum_{j \in [k]} P_j v_j \right) x
\end{align*}
In fact, this shows that $f(x,u)$ is convex in $x$: the first part $x^\top \left( A_0 + \sum_{j \in [k]} P_j u_j \right)^\top \left( A_0 + \sum_{j \in [k]} P_j u_j \right) x$ is a convex quadratic, while the second part $\lambda_{\max}(Y(x))(1 - \|u\|_2^2)$ is a maximum over convex quadratic functions, which is convex. To get the (sub)gradient of the max term, we need to compute the maximizing $v$, which is simply the leading eigenvector of $Y(x)$, denote this by $\bar{v}$. Then we simply take the gradient of $\bar{v}^\top Y(x) \bar{v}$, thus:
\[ \grad_x \lambda_{\max}(Y(x)) = 2 \left( \sum_{j \in [k]} P_j \bar{v}_j \right)^\top \left( \sum_{j \in [k]} P_j \bar{v}_j \right) x. \]
Putting this together, we have
\[ \grad_x f(x,u) = 2 \left( A_0 + \sum_{j \in [k]} P_j u_j \right)^\top \left( A_0 + \sum_{j \in [k]} P_j u_j \right) x + 2 (1 - \|u\|_2^2) \left( \sum_{j \in [k]} P_j \bar{v}_j \right)^\top \left( \sum_{j \in [k]} P_j \bar{v}_j \right) x. \]
Note that if $\|u\|_2 = 1$ (which may often be the case for a subgradient scheme), we need not compute the second term.


\subsection{Summary}

Given $x,u$, and \emph{fixed} data $A_0, P_j \in \bbR^{k \times n}$, we have a robust constraint
\[ \max_{\|u\|_2 \leq 1} \left\{ x^\top \left( A_0 + \sum_{j \in [k]} P_j u_j \right)^\top \left( A_0 + \sum_{j \in [k]} P_j u_j \right) x \right\} \leq c. \]
Defining $Y(x) = \{x^\top P_j^\top P_{j'} x\}_{j,j' \in [k]} \in \bbR^{k \times k}$, the max term can be transformed into 
\begin{align*}
&\max_{\|u\|_2 \leq 1} \left\{ x^\top \left( A_0 + \sum_{j \in [k]} P_j u_j \right)^\top \left( A_0 + \sum_{j \in [k]} P_j u_j \right) x \right\}\\
&= \max_{\|u\|_2 \leq 1} \left\{ x^\top A_0^\top A_0 x + 2 \sum_{j \in [k]} (x^\top A_0^\top P_j x) u_j + u^\top Y(x) u \right\}\\
&= \max_{\|u\|_2 \leq 1} \left\{ x^\top A_0^\top A_0 x + 2 \sum_{j \in [k]} (x^\top A_0^\top P_j x) u_j + u^\top Y(x) u + \lambda_{\max}(Y(x))(1 - \|u\|_2^2) \right\}\\
&= \max_{\|u\|_2 \leq 1} \left\{ x^\top \left( A_0 + \sum_{j \in [k]} P_j u_j \right)^\top \left( A_0 + \sum_{j \in [k]} P_j u_j \right) x + \lambda_{\max}(Y(x))(1 - \|u\|_2^2) \right\}.
\end{align*}

\subsubsection{Subgradients (Ben-Tal et. al. and our approach)}
Here is a recipe for computing the gradients. Given $x \in \bbR^n, u \in \bbR^k$:
\begin{itemize}
\item Compute $Y(x) = \{x^\top P_j^\top P_{j'} x\}_{j,j' \in [k]} \in \bbR^{k \times k}$.
\item Compute $\lambda_{\max}(Y(x))$ and associated leading eigenvector $\bar{v}(x)$.
\item Compute $b(x) = \{x^\top A_0^\top P_j x\}_{j \in [k]} \in \bbR^k$.
\item Compute
\[ \grad_u f(x,u) = 2 b(x) + 2 Y(x) u - 2 \lambda_{\max}(Y(x)) u. \]
\item Compute $A(u) = \left( A_0 + \sum_{j \in [k]} P_j u_j \right)^\top \left( A_0 + \sum_{j \in [k]} P_j u_j \right) \in \bbR^{n \times n}$.
\item If $\|u\|_2 = 1$:
\begin{itemize}
\item Compute
\[ \grad_x f(x,u) = 2 A(u) x. \]
\end{itemize}
Else:
\begin{itemize}
\item Compute $P(\bar{v}(x)) = \left( \sum_{j \in [k]} P_j \bar{v}(x)_j \right)^\top \left( \sum_{j \in [k]} P_j \bar{v}(x)_j \right) \in \bbR^{n \times n}$.
\item Compute
\[ \grad_x f(x,u) = 2 A(u) x + 2(1-\|u\|_2^2) P(\bar{v}(x)) x. \]
\end{itemize}
\end{itemize}

\subsubsection{Computing Lipschitz constants}

We compute the Lipschitz constant $L$ for the function
\[f(x) = x^\top A_0^\top A_0 x + 2 b(x)^\top u + u^\top Y(x) u + \lambda_{\max}(Y(x))(1 - \|u\|_2^2),\]
where $b(x) = Y'(x)^\top A_0 x$ and
\[ Y'(x) = \begin{bmatrix} P_1 x & \ldots & P_k x \end{bmatrix}. \]
Since $f$ is concave in $u$, $L$ is bounded by the norm of the gradient:
\begin{align*}
L &\leq \max_{x \in X, \|u\|_2 \leq 1} \|\grad_u f(x,u)\|_2\\
&= \max_{x \in X, \|u\|_2 \leq 1} \left\| 2 b(x) + 2 (Y(x) - \lambda_{\max}(Y(x)) I_K) u \right\|_2\\
&\leq 2 \max_{x \in X, \|u\|_2 \leq 1} \left\{ \|b(x)\|_2  + \|(Y(x) - \lambda_{\max}(Y(x)) I_K) u\|_2 \right\}\\
&\leq 2 \max_{x \in X} \left\{ \|b(x)\|_2  + \lambda_{\max}(Y(x)) \right\},
\end{align*}
where the last inequality follows since $\|(Y(x) - \lambda_{\max}(Y(x)) I_K) u\|_2 \leq \|Y(x) - \lambda_{\max}(Y(x)) I_K\|_{\Spec} \|u\|_2 \leq \|Y(x) - \lambda_{\max}(Y(x)) I_K\|_{\Spec} \leq \lambda_{\max}(Y(x))$. Notice also that $\lambda_{\max}(Y(x)) = \|Y'(x)\|_{\Spec}^2$ since $Y(x) = Y'(x)^\top Y'(x)$. Now observe that
\begin{align*}
\|b(x)\|_2 &= \| Y'(x)^\top A_0 x \|_2\\
&\leq \|Y'(x)\|_{\Spec} \|A_0 x\|_2\\
&\leq \|Y'(x)\|_{\Spec} \|A_0\|_{\Spec} \|x\|_2.
\end{align*}
We now bound $\|Y'(x)\|_{\Spec}$ as
\begin{align*}
\|Y'(x)\|_{\Spec} &\leq \|Y'(x)\|_{\Fro}\\
&= \sqrt{\sum_{k=1}^K \|P_k x\|_2^2}\\
&\leq  \|x\|_2 \sqrt{\sum_{k=1}^K \|P_k\|_{\Spec}^2}.
\end{align*}
Denote $\sigma = \sqrt{\sum_{k=1}^K \|P_k\|_{\Spec}^2}$ and suppose that $X$ is bounded, so that $\chi = \max_{x \in X} \|x\|_2$, we have
\[ \|Y'(x)\|_{\Spec} \leq \chi \sigma. \]
Putting this all together, we deduce the following bound on the Lipschitz constant:
\[L \leq 2 \chi^2 \sigma \|A_0\|_{\Spec} + 2 \chi^2 \sigma^2 = 2 \chi^2 \sigma \left( \|A_0\|_{\Spec} + \sigma \right). \] 

\subsubsection{Pessimization (Mutapcic and Boyd)}
For a pessimization oracle approach, we solve
\[\max_{\|u\|_2 \leq 1} \left\{ x^\top A_0^\top A_0 x + 2 \sum_{j \in [k]} (x^\top A_0^\top P_j x) u_j + u^\top (Y(x) - \lambda_{\max}(Y(x)) I_k) u + \lambda_{\max}(Y(x))\right\}.\]
To input this into Gurobi for fixed $\bar{x}$, we can just pass the expression
\[\max_{\|u\|_2 \leq 1} \left\{ 2 b(\bar{x})^\top u + u^\top (Y(\bar{x}) - \lambda_{\max}(Y(\bar{x})) I_k) u \right\}.\]
with $u$ variables. Then, once we have a solution $\bar{u}$, we need to `send it to the boundary' by adding multiples of $\bar{v}(\bar{x})$ (the leading eigenvector of $Y(\bar{x})$) until $\|\bar{u} + \alpha \bar{v}(\bar{x})\|=1$. This ensures that $\bar{u} + \alpha \bar{v}(\bar{x})$ is an optimal solution to the noncpncave quadratic as well as the concave relaxation. We then update $\bar{u} := \bar{u} + \alpha \bar{v}(\bar{x})$.

Then, if we want to solve a nominal program for given then $\bar{u}$, we need only solve with constraints
\[ x^\top A(\bar{u}) x \leq c. \]
In other words, eo \emph{don't need to worry about the $\lambda_{\max}(Y(x))$ part}, since we made $\|\bar{u}\|_2 = 1$.




%\bibliographystyle{abbrv}
%\bibliography{bibliography}

\end{document}















